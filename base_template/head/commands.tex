% shortcut => use \abbrev{CMS}{Compact Muon Solenoid}
\let\abbrev\nomenclature%
%
% manual indentation shortcuts => \ind(xs, s, l, xl)
\newcommand{\indxs}[0]{\hspace*{0.38em}}%
\newcommand{\inds}[0]{\hspace*{0.75em}}%
\newcommand{\ind}[0]{\hspace*{1.5em}}%
\newcommand{\indl}[0]{\hspace*{2em}}%
\newcommand{\indxl}[0]{\hspace*{3em}}%
%
% vertical offset shortucts => \pg(s, l)
\newcommand{\pgs}[0]{\vspace*{0.125em}}
\newcommand{\pg}[0]{\vspace*{0.25em}}
\newcommand{\pgl}[0]{\vspace*{0.33em}}
%
% associate figures with sections
\renewcommand{\thefigure}{\thesection.\arabic{figure}}%
%
% associate tables with sections
\renewcommand{\thetable}{\thesection.\arabic{table}}%
%
% associate equations with sections
\renewcommand{\theequation}{\thesection.\arabic{equation}}%
%
% associate listings with sections
\renewcommand{\thelisting}{\thesection.\arabic{listing}}

% reset both counters after each section
\newcommand{\secsep}{\hspace{1em}}%
\newcommand{\Section}[2][]{%
    \FloatBarrier%
    \pageshift
    \thispagestyle{plain}%
    \setcounter{subsection}{0}%
    \setcounter{figure}{0}%
    \setcounter{table}{0}%
    \setcounter{equation}{0}%
    \setcounter{listing}{0}
    \refstepcounter{section}%
    \ifthenelse{\equal{#1}{}}{\section*{\thesection\secsep#2}}{\section*{\thesection\secsep#1}}%
    \sectionmark{#2}%
    \addcontentsline{toc}{section}{\thesection\secsep#2}%
}%
\newcommand{\Subsection}[2][]{%
    \FloatBarrier%
    \setcounter{subsubsection}{0}%
    \refstepcounter{subsection}%
    \ifthenelse{\equal{#1}{}}{\subsection*{\thesubsection\secsep#2}}{\subsection*{\thesubsection\secsep#1}}%
    \subsectionmark{#2}%
    \addcontentsline{toc}{subsection}{\thesubsection\secsep#2}%
}%
\newcommand{\Subsubsection}[2][]{%
    % \FloatBarrier%
    \refstepcounter{subsubsection}%
    \ifthenelse{\equal{#1}{}}{\subsubsection*{\thesubsubsection\secsep#2}}{\subsubsection*{\thesubsubsection\secsep#1}}%
    \subsubsectionmark{#2}%
    \addcontentsline{toc}{subsubsection}{\thesubsubsection\secsep#2}%
}%
%
% shortcut for a simple table
\newcommand{\Table}[6]{%
\begin{table}[#1]%
\begin{center}%
\begin{tabular}{#5}%
#6%
\end{tabular}%
\caption[#3]{#3#4}%
\label{#2}%
\end{center}%
\end{table}%
}%
%
% shortcut for a simple figure
\newcommand{\Figure}[7]{
\begin{figure}[#1]%
\begin{center}%
\includegraphics[width=#2]{#6}
\caption[#4]{#4#5}%
\label{fig:#3}%
\end{center}%
\end{figure}%
}%

\newcommand{\Faynman}[5]{
\vspace{2mm}
\begin{fmffile}{#1}
\begin{fmfgraph*}(#2)
\fmfstraight
\fmfleftn{l}{#3}
\fmfrightn{r}{#4}
#5
\end{fmfgraph*}
\end{fmffile}
\vspace{2mm}
}

% shortcut for a minted code listing
\newcommand{\Mintedfile}[6]{
\begin{listing}[#1]
\begin{center}
\inputminted{#5}{#6}
\caption[#2]{#2#3}
\label{#4}
\end{center}
\end{listing}
}

% optinally adds an empty page between chapters to ensure that new chapters allways start
% on odd pages
\newcommand{\pageshift}{
\FloatBarrier%
\cleardoublepage
% \newpage
% \ifodd\thepage%
% %nothing to do
% \else
% \newpage
% \thispagestyle{empty}
% \null%
% \fi
}%
%
% fast equations using align
\newcommand{\Appendix}[1]{
\newpage%
\thispagestyle{plain}%
\phantomsection%
#1
}

% some math commands
\newcommand{\lag}{\mathcal{L}}
\newcommand{\del}{\partial}
\newcommand{\cdel}{\cancel{\partial}}
\newcommand{\dagg}[2][\mu]{i \gamma^#1 #2_#1}
\newcommand{\vect}[2][pmatrix]{\begin{#1}\mqty{#2}\end{#1}}
\newcommand{\HH}{di-Higgs}
\newcommand{\bref}[1]{(see \cref{#1})}

\newcommand{\s}[1]{\acrlong*{#1}}
\newcommand{\p}[1]{\acrlongpl*{#1}}
\newcommand{\eg}{e.g.\ }
\newcommand{\SF}[1]{SF_\text{#1}}
\newcommand{\pp}[4]{{/home/bfis/mnt/lx3a24/public/analyses/DiHiggs/#1/RunIISummer16MiniAODv2_80X/#2/#3/pdf/#4}.pdf}
