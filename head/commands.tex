% !TEX root = ../thesis.tex
% chktex-file 41

% shortcut => use \abbrev{CMS}{Compact Muon Solenoid}
\let\abbrev\nomenclature%
%
% manual indentation shortcuts => \ind(xs, s, l, xl)
\newcommand{\indxs}[0]{\hspace*{0.38em}}%
\newcommand{\inds}[0]{\hspace*{0.75em}}%
\newcommand{\ind}[0]{\hspace*{1.5em}}%
\newcommand{\indl}[0]{\hspace*{2em}}%
\newcommand{\indxl}[0]{\hspace*{3em}}%
%
% vertical offset shortucts => \pg(s, l)
\newcommand{\pgs}[0]{\vspace*{0.125em}}
\newcommand{\pg}[0]{\vspace*{0.25em}}
\newcommand{\pgl}[0]{\vspace*{0.33em}}
%
% reference shortcuts
\newcommand{\refsec}[1]{sec.~\ref{sec:#1}}
\newcommand{\refssec}[1]{sec.~\ref{ssec:#1}}
\newcommand{\refsssec}[1]{sec.~\ref{sssec:#1}}
\newcommand{\refapp}[1]{app.~\ref{app:#1}}
\newcommand{\reffig}[1]{fig.~\ref{fig:#1}}
\newcommand{\refeq}[1]{eq.~\ref{eq:#1}}
\newcommand{\reftab}[1]{tab.~\ref{tab:#1}}
%
% associate figures with sections
\renewcommand{\thefigure}{\thesection.\arabic{figure}}%
%
% associate tables with sections
\renewcommand{\thetable}{\thesection.\arabic{table}}%
%
% associate equations with sections
\renewcommand{\theequation}{\thesection.\arabic{equation}}%
%
% associate listings with sections
\renewcommand{\thelisting}{\thesection.\arabic{listing}}

% reset both counters after each section
\newcommand{\secsep}{\hspace{1em}}%
\newcommand{\Section}[2][]{%
    \FloatBarrier%
    \pageshift{}
    \thispagestyle{plain}%
    \setcounter{subsection}{0}%
    \setcounter{figure}{0}%
    \setcounter{table}{0}%
    \setcounter{equation}{0}%
    \setcounter{listing}{0}
    \refstepcounter{section}%
    \ifthenelse{\equal{#1}{}}{\section*{\thesection\secsep#2}}{\section*{\thesection\secsep#1}}%
    \sectionmark{#2}%
    \addcontentsline{toc}{section}{\thesection\secsep#2}%
}%
\newcommand{\Subsection}[2][]{%
    \FloatBarrier%
    \setcounter{subsubsection}{0}%
    \refstepcounter{subsection}%
    \ifthenelse{\equal{#1}{}}{\subsection*{\thesubsection\secsep#2}}{\subsection*{\thesubsection\secsep#1}}%
    \subsectionmark{#2}%
    \addcontentsline{toc}{subsection}{\thesubsection\secsep#2}%
}%
\newcommand{\Subsubsection}[2][]{%
    % \FloatBarrier%
    \refstepcounter{subsubsection}%
    \ifthenelse{\equal{#1}{}}{\subsubsection*{\thesubsubsection\secsep#2}}{\subsubsection*{\thesubsubsection\secsep#1}}%
    \subsubsectionmark{#2}%
    \addcontentsline{toc}{subsubsection}{\thesubsubsection\secsep#2}%
}%
%
% shortcut for a simple table
\newcommand{\Table}[6]{%
\begin{table}[#1]%
\begin{center}%
\begin{tabular}{#5}%
#6%
\end{tabular}%
\caption[#3]{#3#4}%
\label{#2}%
\end{center}%
\end{table}%
}%
%
% shortcut for a simple figure
\makeatletter
\define@key{Figure}{pos}{\def\Figure@Pos{#1}}
\define@key{Figure}{lab}{\def\Figure@Lab{#1}}
\define@key{Figure}{cap}{\def\Figure@Cap{#1}}
\define@key{Figure}{shortCap}{\def\Figure@ShortCap{#1}}
\define@key{Figure}{opts}{\def\Figure@Opts{#1}}
\define@key{Figure}{file}{\def\Figure@File{#1}}
\define@key{Figure}{fileExt}{\def\Figure@FileExt{#1}}

\newcommand{\incgfx}[3][pdf]{%
    \IfFileExists{./#3.#1}%
        {\expandafter\includegraphics\expandafter[#2]{#3.#1}}% test file found
        {%
            \IfFileExists{./#3-nonfinal.#1}%
                {
                    \expandafter\includegraphics\expandafter[#2]{#3-nonfinal.#1}%
                    \caption{non-final}%
                }% non-final file version exists
                {%
                    \expandafter\includegraphics\expandafter[#2]{fig/dummy.pdf}%
                    \caption{#3.#1}%
                }% there does not exist anything
        }% test file not found
}
\newcommand{\Figure}[4][]{%
\setkeys{Figure}{pos=ht!,fileExt=pdf,opts={width=\textwidth},shortCap={#3}}%
\setkeys{Figure}{#1}%
% \@ifundefined{Figure@Pos}{\def\Figure@Pos{ht!}}{}%
% \@ifundefined{Figure@FileExt}{\def\Figure@FileExt{pdf}}{}%
% \@ifundefined{Figure@Opts}{\def\Figure@Opts{width=\textwidth}}{}%
% \@ifundefined{Figure@ShortCap}{\def\Figure@ShortCap{#3}}{}%
\expandafter\figure\expandafter[\Figure@Pos]%
\begin{center}%
\incgfx[\Figure@FileExt]{\Figure@Opts}{#2}%
\caption[\Figure@ShortCap]{#3}%
\label{fig:#4}%
\end{center}%
\endfigure%
}%
\makeatother

\newcommand{\Faynman}[5]{
\vspace{2mm}
\begin{fmffile}{#1}
\begin{fmfgraph*} (#2)
\fmfstraight{}
\fmfleftn{l}{#3}
\fmfrightn{r}{#4}
#5
\end{fmfgraph*}
\end{fmffile}
\vspace{2mm}
}

% shortcut for a minted code listing
\newcommand{\Mintedfile}[6]{
\begin{listing}[#1]
\begin{center}
\inputminted{#5}{#6}
\caption[#2]{#2#3}
\label{#4}
\end{center}
\end{listing}
}

% optinally adds an empty page between chapters to ensure that new chapters allways start
% on odd pages
\newcommand{\pageshift}{
\FloatBarrier%
\cleardoublepage{}
% \newpage
% \ifodd\thepage%
% %nothing to do
% \else
% \newpage
% \thispagestyle{empty}
% \null%
% \fi
}%
%
% fast equations using align
\newcommand{\Appendix}[1]{
\newpage%
\thispagestyle{plain}%
\phantomsection%
#1
}

% name formatting substitutions
\newcommand{\pao}{Pierre Auger Observatory}
\newcommand{\geant}{\textit{GEANT4}}
\newcommand{\pythiaDelphes}{\textit{Pythia \& Delphes}}
\newcommand{\pythia}{\textit{Pythia}}
\newcommand{\delphes}{\textit{Delphes}}

% High energy physics
\DeclareSIUnit\annum{a}
\DeclareSIUnit\micron{\micro\metre}
\DeclareSIUnit\mrad{\milli\rad}
\DeclareSIUnit\gauss{G}
\DeclareSIUnit\parsec{pc}
\DeclareSIUnit\eVperc{\eV\per\clight}
\DeclareSIUnit\nanobarn{\nano\barn}
\DeclareSIUnit\picobarn{\pico\barn}
\DeclareSIUnit\femtobarn{\femto\barn}
\DeclareSIUnit\attobarn{\atto\barn}
\DeclareSIUnit\zeptobarn{\zepto\barn}
\DeclareSIUnit\yoctobarn{\yocto\barn}
\DeclareSIUnit\nb{\nano\barn}
\DeclareSIUnit\pb{\pico\barn}
\DeclareSIUnit\fb{\femto\barn}
\DeclareSIUnit\ab{\atto\barn}
\DeclareSIUnit\zb{\zepto\barn}
\DeclareSIUnit\yb{\yocto\barn}

% some math commands
\DeclareMathOperator\arctanh{arctanh}
\DeclareMathOperator\sgn{sgn}
\DeclareMathOperator\relu{relu}
\DeclareMathOperator\softmax{softmax}
\DeclareMathOperator*{\argmin}{arg\,min}
\DeclareMathOperator*{\argmax}{arg\,max}
\DeclareMathOperator\Var{Var}
\DeclareMathOperator\E{E}

\newcommand{\iu}{\mathrm{i}\mkern1mu}
\newcommand{\loss}[1]{\mathcal{L}_{#1}}
\newcommand{\del}{\partial}
\newcommand{\cdel}{\cancel{\partial}}
\newcommand{\dagg}[2][\mu]{i \gamma^#1 #2_#1}
\newcommand{\vect}[2][pmatrix]{\begin{#1}\mqty{#2}\end{#1}}
\newcommand{\HH}{di-Higgs}
\newcommand{\bref}[1]{(see \cref{#1})}

\newcommand{\s}[1]{\acrlong*{#1}}
\newcommand{\p}[1]{\acrlongpl*{#1}}
\newcommand{\eg}{e.g.\ }
\newcommand{\SF}[1]{SF_\text{#1}}
\newcommand{\pp}[4]{{/home/bfis/mnt/lx3a24/public/analyses/DiHiggs/#1/RunIISummer16MiniAODv2_80X/#2/#3/pdf/#4}.pdf}


% particle zoo graph
% Standard model of physics
% Author: Carsten Burgard; Edit: Dennis Noll
\tikzset{%
        brace/.style = { decorate, decoration={brace, amplitude=5pt} },
       mbrace/.style = { decorate, decoration={brace, amplitude=5pt, mirror} },
        label/.style = { black, midway, scale=0.5, align=center },
     toplabel/.style = { label, above=.5em, anchor=south },
    leftlabel/.style = { label,rotate=-90,left=.5em,anchor=north },
  bottomlabel/.style = { label, below=.5em, anchor=north },
        force/.style = { rotate=-90,scale=0.4 },
        round/.style = { rounded corners=2mm },
       legend/.style = { right,scale=0.4 },
        nosep/.style = { inner sep=0pt },
   generation/.style = { anchor=base }
}

\newcommand\particle[7][white]{%
  \begin{tikzpicture}[x=1cm, y=1cm]
    \path[fill=#1,blur shadow={shadow blur steps=5}] (0.1,0) -- (0.9,0)
        arc (90:0:1mm) -- (1.0,-0.9) arc (0:-90:1mm) -- (0.1,-1.0)
        arc (-90:-180:1mm) -- (0,-0.1) arc(180:90:1mm) -- cycle;
    \ifstrempty{#7}{}{\path[fill=purple!50!white]
        (0.6,0) --(0.7,0) -- (1.0,-0.3) -- (1.0,-0.4);}
    \ifstrempty{#6}{}{\path[fill=green!50!black!50] (0.7,0) -- (0.9,0)
        arc (90:0:1mm) -- (1.0,-0.3);}
    \ifstrempty{#5}{}{\path[fill=orange!50!white] (1.0,-0.7) -- (1.0,-0.9)
        arc (0:-90:1mm) -- (0.7,-1.0);}
    \draw[\ifstrempty{#2}{dashed}{black}] (0.1,0) -- (0.9,0)
        arc (90:0:1mm) -- (1.0,-0.9) arc (0:-90:1mm) -- (0.1,-1.0)
        arc (-90:-180:1mm) -- (0,-0.1) arc(180:90:1mm) -- cycle;
    \ifstrempty{#7}{}{\node at(0.825,-0.175) [rotate=-45,scale=0.2] {#7};}
    \ifstrempty{#6}{}{\node at(0.9,-0.1)  [nosep,scale=0.17] {#6};}
    \ifstrempty{#5}{}{\node at(0.9,-0.9)  [nosep,scale=0.2] {#5};}
    \ifstrempty{#4}{}{\node at(0.1,-0.1)  [nosep,anchor=west,scale=0.25]{#4};}
    \ifstrempty{#3}{}{\node at(0.1,-0.85) [nosep,anchor=west,scale=0.3] {#3};}
    \ifstrempty{#2}{}{\node at(0.1,-0.5)  [nosep,anchor=west,scale=1.5] {#2};}
  \end{tikzpicture}
}

\newcommand{\drawparticlezoo}{
    \centering
    \resizebox{0.9\textwidth}{!}{

    \begin{tikzpicture}[x=1.2cm, y=1.2cm]
      \draw[round] (-0.5,0.5) rectangle (3.6,-1.5);
      \draw[round] (-0.6,0.6) rectangle (4.0,-2.5);
      \draw[round] (-0.7,0.7) rectangle (4.5,-3.5);

      \node at(0, 0)   {\particle[gray!20!white]
                       {$u$}        {up}       {$2.3$ MeV}{1/2}{$2/3$}{ }};
      \node at(0,-1)   {\particle[gray!20!white]
                       {$d$}        {down}    {$4.8$ MeV}{1/2}{$-1/3$}{ }};
      \node at(0,-2)   {\particle[gray!20!white]
                       {$e$}        {electron}       {$511$ keV}{1/2}{$-1$}{}};
      \node at(0,-3)   {\particle[gray!20!white]
                       {$\nu_e$}    {$e$~neutrino}         {$<2$ eV}{1/2}{}{}};
      \node at(1, 0)   {\particle
                       {$c$}        {charm}   {$1.28$ GeV}{1/2}{$2/3$}{ }};
      \node at(1,-1)   {\particle
                       {$s$}        {strange}  {$95$ MeV}{1/2}{$-1/3$}{ }};
      \node at(1,-2)   {\particle
                       {$\mu$}      {muon}         {$105.7$ MeV}{1/2}{$-1$}{}};
      \node at(1,-3)   {\particle
                       {$\nu_\mu$}  {$\mu$~neutrino}    {$<190$ keV}{1/2}{}{}};
      \node at(2, 0)   {\particle
                       {$t$}        {top}    {$173.2$ GeV}{1/2}{$2/3$}{ }};
      \node at(2,-1)   {\particle
                       {$b$}        {bottom}  {$4.7$ GeV}{1/2}{$-1/3$}{ }};
      \node at(2,-2)   {\particle
                       {$\tau$}     {tau}          {$1.777$ GeV}{1/2}{$-1$}{}};
      \node at(2,-3)   {\particle
                       {$\nu_\tau$} {$\tau$~neutrino}  {$<18.2$ MeV}{1/2}{}{}};
      \node at(3,-3)   {\particle[orange!20!white]
                       {$W^{\hspace{-.3ex}\scalebox{.5}{$\pm$}}$}
                                    {}              {$80.4$ GeV}{1}{$\pm1$}{}};
      \node at(4,-3)   {\particle[orange!20!white]
                       {$Z$}        {}                    {$91.2$ GeV}{1}{}{}};
      \node at(3,-2) {\particle[green!50!black!20]
                       {$\gamma$}   {photon}                        {}{1}{}{}};
      \node at(3,-1) {\particle[purple!20!white]
                       {$g$}        {gluon}                    {}{1}{}{ }};
      \node at(5.0,0)    {\particle[gray!50!white]
                       {$H$}        {Higgs}              {$125.1$ GeV}{0}{}{}};

      \node at(3.7,-0.5) [force]      {strong force (color)};
      \node at(4.1,-1.5) [force]    {electromagnetic force (charge)};
      \node at(4.6,-2.4) [force] {weak force (weak isospin)};

      \draw [<-] (2.5,0.3)   -- (2.7,0.3)          node [legend] {charge};
      \draw [<-] (2.5,0.15)  -- (2.7,0.15)         node [legend] {color charged};
      \draw [<-] (2.05,0.25) -- (2.3,0) -- (2.7,0) node [legend]   {mass};
      \draw [<-] (2.5,-0.3)  -- (2.7,-0.3)         node [legend]   {spin};

      \draw [mbrace] (-0.8,0.5)  -- (-0.8,-1.5)
                     node[leftlabel] {6 quarks\\(+6 anti-quarks)};
      \draw [mbrace] (-0.8,-1.5) -- (-0.8,-3.5)
                     node[leftlabel] {6 leptons\\(+6 anti-leptons)};
      \draw [mbrace] (-0.5,-3.6) -- (2.5,-3.6)
                     node[bottomlabel]
                     {12 fermions\\(+12 anti-fermions)\\increasing mass $\to$};
      \draw [mbrace] (2.5,-3.6) -- (5.5,-3.6)
                     node[bottomlabel] {13 bosons\\(eight gluons)};

      \draw [brace] (-0.5,.8) -- (0.5,.8) node[toplabel]         {standard matter};
      \draw [brace] (0.5,.8)  -- (2.5,.8) node[toplabel]         {unstable matter};
      \draw [brace] (2.5,.8)  -- (4.5,.8) node[toplabel]          {force carriers};
      \draw [brace] (4.5,.8)  -- (5.5,.8) node[toplabel]       {Higgs\\sector};

      \node at (0,1.2)   [generation] {1\tiny st};
      \node at (1,1.2)   [generation] {2\tiny nd};
      \node at (2,1.2)   [generation] {3\tiny rd};
      \node at (2.8,1.2) [generation] {\tiny generation};
    \end{tikzpicture}
    }
}
